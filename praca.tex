\documentclass[licencjacka]{pracamgr}

\usepackage{polski}
\usepackage[utf8]{inputenc}

% Dane magistranta:

\author{Autorzy}

\nralbumu{Albumy}

\title{Super tajny projekt dla ZUSu\\
	(a tak naprawdę dla prezesa)}


\tytulang{Nicht sprechen englisch}

%kierunek: Matematyka, Informatyka, ...
\kierunek{Informatyka}

% informatyka - nie okreslamy zakresu (opcja zakomentowana)
% matematyka - zakres moze pozostac nieokreslony,
% a jesli ma byc okreslony dla pracy mgr,
% to przyjmuje jedna z wartosci:
% {metod matematycznych w finansach}
% {metod matematycznych w ubezpieczeniach}
% {matematyki stosowanej}
% {nauczania matematyki}
% Dla pracy licencjackiej mamy natomiast
% mozliwosc wpisania takiej wartosci zakresu:
% {Jednoczesnych Studiow Ekonomiczno--Matematycznych}

% \zakres{Tu wpisac, jesli trzeba, jedna z opcji podanych wyzej}

% Praca wykonana pod kierunkiem:
% (podać tytuł/stopień imię i nazwisko opiekuna
% Instytut
% ew. Wydział ew. Uczelnia (jeżeli nie MIM UW))
\opiekun{Prezes\\
  ZUS\\
  }

% miesiąc i~rok:
\date{Listopad 2016}

%Podać dziedzinę wg klasyfikacji Socrates-Erasmus:
\dziedzina{ 
%11.0 Matematyka, Informatyka:\\ 
%11.1 Matematyka\\ 
%11.2 Statystyka\\ 
11.3 Informatyka\\ 
%11.4 Sztuczna inteligencja\\ 
%11.5 Nauki aktuarialne\\
%11.9 Inne nauki matematyczne i informatyczne
}

%Klasyfikacja tematyczna wedlug AMS (matematyka) lub ACM (informatyka)
\klasyfikacja{D. Software\\
  D.127. Ubezpieczenia społeczne}

% Słowa kluczowe:
\keywords{lenistwo stosowane, spiseg, trudne sprawy}

% Tu jest dobre miejsce na Twoje własne makra i~środowiska:
\newtheorem{defi}{Definicja}[section]

% koniec definicji

\begin{document}
\maketitle

%tu idzie streszczenie na strone poczatkowa
\begin{abstract}
  Potem dopiszemy.
\end{abstract}

\tableofcontents
%\listoffigures
%\listoftables

\chapter*{Wprowadzenie} 
\addcontentsline{toc}{chapter}{Wprowadzenie}
Tutaj napiszemy jakiś wstęp, nie mam pojęcia o czym. Janusz mówi: ,,Wstęp (a czasem nie tylko wstęp): opis całego projektu, w naturalny sposób daje się tu wykorzystać fragmenty wizji. Dyskusja rozwiązań konkurencyjnych - w zależności od projektu: na świecie, w Polsce, w zamawiającej firmie - ze śladem w bibliografii. To może być część wstępu, może być też osobny rozdział. "

\chapter{Wizja}\label{r:wizja}
Główna część pracy. Jeden bądź kilka rozdziałów poświęconych osiągnięciom, problemom, sukcesom projektu. To jest to miejsce, gdzie należy się pochwalić tym, co Państwo osiągnęli. Jeśli Państwo się sami nie pochwalicie, nikt nigdy nie będzie wiedział ile pracy, pomysłowości i wysiłku kosztowało zrealizowanie projektu. Chwalić się można i należy różnymi rzeczami.

Osiągnięciami (projekt ma świetny interfejs; projekt jako pierwszy rozwiązuje jakiś problem; użytkownicy projektu są bardzo zadowoleni, a klient zawarł z Państwem umowę na przyszłość; Państwa firma właśnie wchodzi na New Connect; 10\% z pierwszych zarobionych milionów przeznaczyli Państwo na fundację wspierania UW; itp.).
Ciekawymi rozwiązaniami i dobrym kodem (projekt stworzono od początku do końca wg nowoczesnej metodologii; w projekcie użyto zaawansowanych narzędzi np. do recenzji kodu; projekt korzysta z nowoczesnych bibliotek; w projekcie zastosowano niebanalne algorytmy; projekt używa dedykowanego sprzętu; projekt korzysta z rozbudowanej infrastruktury: serwer WWW + serwer b.d. + serwer aplikacji + serwer poczty + dwa mendle innych ważnych serwerów; itp.).
Włożoną w projekt pracą (dyskusja innych możliwych rozwiązań odrzucanych na kolejnych etapach pracy; opis problemów z biblioteką pobraną z internetu, która wg dokumentacji miała działać, a na skutek jej błędów uruchomienie systemu zajęło miesiąc dodatkowej pracy; problemów z zamawiającym, którego poszczególni przedstawiciele uparli się składać sprzeczne między sobą wymagania; opis organizacji pracy zespołu, w którym wszyscy członkowie mieszkają każdy w innym mieście poza Warszawą, studiują na różnych kierunkach i dotąd się nie znali, jednocześnie pracują i to każdy w innych godzinach, a na koniec dwójka z nich się pobrała, ma trojaczki, właśnie wzięła urlop dziekański i chwilowo podróżuje tratwą po Pacyfiku; itp.).

Zwykle warto dodać jakiś kluczowy przypadek użycia z Państwa dokumentacji. W zależności od projektu mogą być bardzo pożądane zrzuty ekranów kluczowych części projektu. Znów w zależności od projektu mogą być potrzebne fragmenty kodu (do kilkunastu wierszy bezpośrednio w kodzie, do 2 stron jako załączniki do treści dokumentu).
\chapter{Prototyp}\label{r:prototyp}

\chapter{Wkład poszczególnych członków zespołu w projekt}\label{r:wklad}
Opis tego co poszczególne osoby zrobiły w ramach projektu. To bardzo ważne, proszę zapisać jako osobny rozdział (czyli np. nie podrozdział).

\chapter{Spis zawartości dołączonej płyty CD}\label{r:spis}
Dokładny spis zawartości towarzyszącej płytki (p. dalej). To bardzo ważne, proszę zapisać jako osobny rozdział (czyli np. nie podrozdział). Płytka CD/DVD/Blu-ray/...

Zawiera:\\
Pełną dokumentacją projektu w łatwo dającym się odczytać formacie (najlepiej pdf + źródło).
Program (w postaci źródłowej i potencjalnie umożliwiającej uruchomienie, to może oznaczać np. dostarczenie stosownych plików makefile, pomocniczych plików z danymi, opisu instalacji itp.).
Wszelkie inne dokumenty powstałe podczas zajęć (np. teksty prezentacji, teksty pracy z poprzedniego dużego punktu, itp.).

Płyta jest częścią pracy - trzeba tyle płyt co drukowanych egzemplarzy pracy. Płytkę trzeba przymocować do pracy, tak by a) nie wypadała b) dało się ją wyjąć i odczytać w komputerze :).

\begin{thebibliography}{99}
\addcontentsline{toc}{chapter}{Bibliografia}

\bibitem[WWW]{internet} Autor nieznany, \textit{Dokumentacja django}.

\end{thebibliography}

\end{document}


%%% Local Variables:
%%% mode: latex
%%% TeX-master: t
%%% coding: latin-2
%%% End:
