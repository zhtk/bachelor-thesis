\documentclass[licencjacka]{pracamgr}

\usepackage{polski}
\usepackage[utf8]{inputenc}

% Dane magistranta:

\author{Autorzy}

\nralbumu{Albumy}

\title{Dużo mikrousług}

\tytulang{Project}

\kierunek{Informatyka}

% informatyka - nie okreslamy zakresu (opcja zakomentowana)
% \zakres{Tu wpisac, jesli trzeba, jedna z opcji podanych wyzej}

% Praca wykonana pod kierunkiem:
% (podać tytuł/stopień imię i nazwisko opiekuna
% Instytut
% ew. Wydział ew. Uczelnia (jeżeli nie MIM UW))
\opiekun{mgr Michał Możdżonek\\
  ZUS\\
  }

% miesiąc i~rok:
\date{Listopad 2016}

%Podać dziedzinę wg klasyfikacji Socrates-Erasmus:
\dziedzina{ 
%11.0 Matematyka, Informatyka:\\ 
%11.1 Matematyka\\ 
%11.2 Statystyka\\ 
11.3 Informatyka\\ 
%11.4 Sztuczna inteligencja\\ 
%11.5 Nauki aktuarialne\\
%11.9 Inne nauki matematyczne i informatyczne
}

%Klasyfikacja tematyczna wedlug AMS (matematyka) lub ACM (informatyka)
\klasyfikacja{D. Software\\
  D.127. Ubezpieczenia społeczne}

% Słowa kluczowe:
\keywords{mikrousługi, SOA, trudne sprawy}

% Tu jest dobre miejsce na Twoje własne makra i~środowiska:
\newtheorem{defi}{Definicja}[section]

% koniec definicji

\begin{document}
\maketitle

%tu idzie streszczenie na strone poczatkowa
\begin{abstract}
  Potem dopiszemy.
\end{abstract}

\tableofcontents
%\listoffigures
%\listoftables

\chapter*{Wprowadzenie} 
\addcontentsline{toc}{chapter}{Wprowadzenie}
Tutaj napiszemy jakiś wstęp, nie mam pojęcia o czym. Janusz mówi: ,,Wstęp (a czasem nie tylko wstęp): opis całego projektu, w naturalny sposób daje się tu wykorzystać fragmenty wizji. Dyskusja rozwiązań konkurencyjnych - w zależności od projektu: na świecie, w Polsce, w zamawiającej firmie - ze śladem w bibliografii. To może być część wstępu, może być też osobny rozdział. "

\section{Uwagi Janusza}
Główna część pracy. Jeden bądź kilka rozdziałów poświęconych osiągnięciom, problemom, sukcesom projektu. To jest to miejsce, gdzie należy się pochwalić tym, co Państwo osiągnęli. Jeśli Państwo się sami nie pochwalicie, nikt nigdy nie będzie wiedział ile pracy, pomysłowości i wysiłku kosztowało zrealizowanie projektu. Chwalić się można i należy różnymi rzeczami.

Osiągnięciami (projekt ma świetny interfejs; projekt jako pierwszy rozwiązuje jakiś problem; użytkownicy projektu są bardzo zadowoleni, a klient zawarł z Państwem umowę na przyszłość; Państwa firma właśnie wchodzi na New Connect; 10\% z pierwszych zarobionych milionów przeznaczyli Państwo na fundację wspierania UW; itp.).
Ciekawymi rozwiązaniami i dobrym kodem (projekt stworzono od początku do końca wg nowoczesnej metodologii; w projekcie użyto zaawansowanych narzędzi np. do recenzji kodu; projekt korzysta z nowoczesnych bibliotek; w projekcie zastosowano niebanalne algorytmy; projekt używa dedykowanego sprzętu; projekt korzysta z rozbudowanej infrastruktury: serwer WWW + serwer b.d. + serwer aplikacji + serwer poczty + dwa mendle innych ważnych serwerów; itp.).
Włożoną w projekt pracą (dyskusja innych możliwych rozwiązań odrzucanych na kolejnych etapach pracy; opis problemów z biblioteką pobraną z internetu, która wg dokumentacji miała działać, a na skutek jej błędów uruchomienie systemu zajęło miesiąc dodatkowej pracy; problemów z zamawiającym, którego poszczególni przedstawiciele uparli się składać sprzeczne między sobą wymagania; opis organizacji pracy zespołu, w którym wszyscy członkowie mieszkają każdy w innym mieście poza Warszawą, studiują na różnych kierunkach i dotąd się nie znali, jednocześnie pracują i to każdy w innych godzinach, a na koniec dwójka z nich się pobrała, ma trojaczki, właśnie wzięła urlop dziekański i chwilowo podróżuje tratwą po Pacyfiku; itp.).

Zwykle warto dodać jakiś kluczowy przypadek użycia z Państwa dokumentacji. W zależności od projektu mogą być bardzo pożądane zrzuty ekranów kluczowych części projektu. Znów w zależności od projektu mogą być potrzebne fragmenty kodu (do kilkunastu wierszy bezpośrednio w kodzie, do 2 stron jako załączniki do treści dokumentu).

\chapter{Organizacja pracy}

Pierwszym poważnym wyzwaniem, przed którym staneliśmy, była organizacja pracy zespołu. Każdy członek zespołu
pochodzi z innego miasta i uczęszcza na inne zajęcia, ponadto część członków pracuje. Znalezienie pory i dnia
tygodnia, w którym moglibyśmy się spotykać, stanowiło spore wyzwanie. W gospodarowaniu czasem i ustalaniu terminów
spotkań z klientem bardzo nam pomogło doodle. % Mądrze rozwinąć!
Komunikacja między członkami zespołu odbywała się głównie przez internet, przy pomocy facebooka %i slacka.
%Projekt prowadziliśmy w lekko zmodyfikowanej zwinnej metodyce opracowanej specjalnie na potrzeby naszego klienta.

\chapter{Wizja systemu}\label{r:wizja}

Celem naszego projektu było stworzenie prototypu internetowego frameworka do komunikacji obywatela z urzędami
podobnego do Platformy Usług Elektronicznych ZUS \cite{zuspue}, a następnie na jego bazie zaimplementować jeden
wybrany przypadek użycia, np. usługę dokumentowania okresów zatrudnienia \cite{zusdoz}. Tym, co wyróżnia nasze
rozwiązanie od innych systemów stosowanych w administracji, jest całkowicie modularna architektura. To pozwala nam
stosunkowo małym kosztem podłączać i odłączać kolejne usługi wraz z rozwojem cyfrowej administracji. Pewną
inspiracją do stworzenia takiego systemu był rządowy portal obywatel.gov.pl \cite{mcobywatel}.

Naszym głównym zadaniem było zaprojektowanie serwisu, który byłby maksymalnie prosty w utrzymaniu i rozwoju, oraz
który łatwo można było skalować. Do którego rozwoju nie trzeba by było zatrudniać doskonale wykształconych
programistów z MIM UW, ale także osoby które kończyły szkoły w mniejszych miastach. System powinien być odporny
na różne błędy programistyczne i spełniać normy 12 Factor App \cite{tfa}, w szczególności być odporny na awarię
pojedyńczej usługi, bez prowadzenia do rozspójnienia danych.

W związku z tym, że projektowaliśmy system dla administracji państwowej z którego będą korzystać miliony obywateli,
musieliśmy bardzo ostrożnie dobierać technologie, z których korzystaliśmy. Wybranie rozwiązania opartego na
niekorzystnej licencji mogło się w przyszłości srogo zemścić. Innym wyzwaniem, które należało uwzględnić w fazie
projektowania, był szeroki zakres użytkowników systemu: od zwykłych obywateli, poprzez urzędników, aż po
przedsiębiorców.

%Elementy, które powinny się znaleźć w dokumencie opisującym wizję to:
%Opis rozwiązania, Korzyści biznesowe, Wizja projektu i zakres, Analiza korzyści, Wymagania biznesowe, Wymagania użytkowników, Wymagania operacyjne, Wymagania systemowe, Wymagania bezpieczeństwa, Zakres projektu, Cele biznesowe, Cele techniczne, Kryteria operacyjne, Ograniczenia, Definicja profili użytkowników, Analizy użytkowania, Scenariusze wykorzystania

\chapter{Prototyp}\label{r:prototyp}

\section{Zarys architektury}
Monolityczna vs mikrousługi, zarysowany podział na tiery.
3 warstwy mikrousług: biznesowa (realizuje logikę i jest dostępna z zewnątrz),
techniczna (niedostępna z zewnątrz, pracuje na rzecz warstwy biznesowej i służy np. do komunikacji
z innymi systemami), bazodanowa.

\section{Problemy z współbieżnością mikrousług}

Nietrudno wyobrazić sobie sytuację, w której dwie współbieżnie działające mikrousługi
wprowadzają konfliktujące ze sobą zmiany. Pewną formą zabezpieczenia przed takimi sytuacjami
mogłoby być rozwiązywanie konfliktów przez mikrousługę pośredniczącą w dostępie do bazy danych.
Taka mikrousługa stanowiłaby wtedy coś w rodzaju sekcji krytycznej, uniemożliwiającej jednoczesne zapisy
i niespójne odczyty (niezależnie od atomowości operacji na bazie danych). 
Niestety, takie podejście może nie wystarczać, np. w przypadku dwóch niezależnie działających usług, które wpierw
odczytują jakąś informację z bazy danych, a następnie na podstawie uzyskanej informacji wybierają rodzaj
akcji do podjęcia i nadpisują część bazy danych.

Jest kilka możliwych rozwiązań tego problemu. Możemy je pogrupować ze względu na następujące cechy:
\begin{itemize}
	\item miejsce przechowywania blokad
	\item sposób zakładania blokad na usługi
	\item wielkość obiektów, które będą blokowane
\end{itemize}
Blokady mogą być przechowywane w dedykowanej, wydzielonej mikrousłudze lub w mikrousłudze, której dotyczą. W naszej opinii
pierwsze podejście jest o tyle lepsze, że pozwala na wykrywanie zakleszczeń. Wadą takiego rozwiązania jest
stosunkowo słabe zrównoleglanie i fakt, że taka mikrousługa stanowiłaby najsłabszy punkt systemu, którego awaria
skutkowałaby zablokowaniem wszystkich pozostałych usług. Blokady mogą być zakładane pojedyńczo, w miarę
postępu transakcji lub na jej samym początku (w takim przypadku mikrousługa musi dokładnie wiedzieć, z jakich
zasobów ma zamiar skorzystać). Wydaje nam się, że najsensowniejszym wariantem jest wykorzystanie rozwiązań
stosowanych w relacyjnych bazach danych, gdzie blokady są zakładane w miarę postępu transakcji, a w przypadku
wykrycia zakleszczenia zmiany są wycofywane, sama zaś usługa wywłaszczana. Przy kolejnej próbie takiej mikrousłudze
nadany zostaje większy priorytet, który zapewnia żywotność. W przypadku implementacji takiego podejścia należy
pamiętać o zapewnieniu mechanizmów wywłaszczania mikrousług, księgowania zmian, śledzenia transakcji, wykrywania
zakleszczeń, nadawania priorytetów oraz o możliwości zakładania możliwie małych blokad na pojedyńcze rekordy.
Niestety, z uwagi na ograniczone zasoby nie byliśmy w stanie zaimplementować mechanizmu synchronizacji w
przedstawiony wyżej sposób. Postanowiliśmy ograniczyć się do muteksów zakładanych na początku transakcji.

Do ustalenia pozostała tylko wielkość blokowanych przez muteks danych. Ze względów wydajnościowych blokowanie
całych mikrousług zapewniających dostęp do danych nie było możliwe. Z uwagi na opóźnienia w przesyłaniu informacji
przez sieć prowadziłoby to z jednej strony do marnowania mocy obliczeniowej serwera bazy danych, a z drugiej strony 
znacząco ograniczałoby ilość jednoczesnych operacji. Z kolei zakładanie blokad na konkretne wiersze w poszczególnych
mikrousługach jest nierealizowalne, gdyż w chwili zakładania muteksu usługa może nie wiedzieć, jakie konkretnie
wiersze zmodyfikuje. Postanowiliśmy znaleźć złoty środek i zakładać blokady na właściciela danych. Wynika to z
faktu, że w typowej urzędniczej praktyce wykonywane operacje dotyczą co najwyżej kilku osób.

Uzbrojeni w powyższe informacje wytypowaliśmy dwie usługi mogące działać jako zarządca rozproszonego muteksa:
redisson \cite{redisson} i ZooKeeper \cite{zookeeper}. Redisson jest rozwiązaniem implementującym algorytm Redlock
\cite{redislock}. Jego najważniejszą cechą jest odporność na awarie usług poprzez automatyczne zwalnianie blokady
po upływie określonego czasu. W internecie ukazała się bardzo interesująca polemika do przytoczonego algorytmu \cite{redisbad}.
Głównym zarzutem czynionym wobec algorytmu Redlock jest brak mechanizmu, który zapobiegałby następującej sytuacji:
mikrousługa A zakłada blokadę na zasób Z. Następnie zaczyna operację nadpisywania danych. W trakcie operacji
działanie mikrousługi z losowego powodu zostaje zawieszone (np. wystąpiło znaczne opóźnienie w połączeniu sieciowym
lub włączył się odśmiecacz pamięci). W tym czasie termin ważności blokady mija, zostaje ona zdjęta, a do akcji
wkracza mikrousługa B, która zakłada swoją blokadę, nadpisuje dane utworzone przez usługę A i kończy swoje
działanie. Na koniec usługa A zostaje wybudzona i kontynuuje swoje działanie, prowadząc do rozspójnienia danych.
Według autora polemiki rozwiązaniem tego problemu jest skorzystanie z mechanizmu współdzielonej blokady
zaimplementowanej w frameworku Apache Curator \cite{curatorlock} i korzystającej pod spodem z Apache ZooKeepera.
Jest to zdecydowanie bezpieczniejsze rozwiązanie, które w przypadku awarii mikrousługi i utraty połączenia
automatycznie zdejmuje blokadę. Ponadto, jest to stosunkowo nowoczesny projekt wykorzystywany i rozwijany przez
m. in. firmę Netflix, co daje pewne gwarancje dotyczące stabilności i przyszłości tego projektu.

Jest jeszcze jeden problem, na który użytkownicy naszej platformy mogą się natknąć. Mianowicie, nasza platforma
może integrować poprzez różne adaptery już istniejące usługi, które zupełnie nie są świadome istnienia
zaproponowanego mechanizmu blokad, więc nie są odporne na błędy wynikające z współbieżności.
Projektanci mikrousług korzystających z naszej platformy muszą być świadomi wszystkiego, co wyżej napisano.
Aby maksymalnie ułatwić im pracę uwzględniliśmy wyniki powyższych rozważań  przy projektowaniu API, jednak nie
jesteśmy w stanie rozwiązać wszystkich pojawiających się problemów.

\section{Zapewnienie transakcyjności}

Jednym z wymagań pozafunkcjonalnych było zapewnienie skalowalności serwisu i mikrousług w sposób zachowujący transakcyjność.
Dołożenie lub utrata serwera nie powinna wpływać na poprawność transakcji i nie powinna w żadnym przypadku
prowadzić do utraty lub rozspójnienia danych. Typowym rozwiązaniem stosowanym w takim przypadku jest prowadzenie
komunikacji między klientem a serwerem w sposób bezstanowy.

Powstaje tutaj pytanie, czy w przypadku naszego serwisu jest to możliwe? Dopuszczalny jest scenariusz, w którym
mikrousługa pyta użytkownika o jakąś informację (np. stan cywilny), a następnie w zależności od udzielonej odpowiedzi
prosi o dodatkowe informacje, np. o dane małżonka. Rozwiązaniem tego problemu jest przeniesienie stanowości
transakcji z serwera do klienta. Wszystkie informacje dotyczące transakcji będą trzymane po stronie klienta, a
operacje, które mogą zaburzać integralność danych będą wykonywane w jednym momencie i w całości, bez rozbijania
na wiele części. Dane przechowywane w kliencie przez poszczególne mikrousługi będą odseparowane między sobą, tak aby
zapobiec konfliktom. W uzasadnionych przypadkach możliwe byłoby przechowywanie części danych (takich jak duże pliki)
po stronie serwera, oraz trzymanie w kliencie ich identyfikatora.

\section{Przegląd technologii do tworzenia frontendu}
Django, PHP, Spring i inne serwerowe podejścia są be. Zastanawiamy się nad dojo vs angular 2 vs react (odpada, słaba licencja).

\section{Szyny danych}
WebMethods vs WSO2, vs Talend, Mule vs zato.io, a może bez żadnej szyny usług (single point of failure, szyny lubią się zapychać)?

\section{Usługi monitorowania stanu mikrousług}
Apache zookeeper vs Netflix eureka vs ...

\section{Przegląd baz danych}
Postgres + pg\_shard i mongo db. Sensowne jest też CouchDB. Redis jest zbyt prymitywny, a jego obsługa transakcji zbyt ograniczona, chociaż
może się przydać przy synchronizacji \cite{redislock}.

\section{API udostępniane przez mikrousługi i frontend}

\chapter{Wkład poszczególnych członków zespołu w projekt}\label{r:wklad}
Opis tego co poszczególne osoby zrobiły w ramach projektu. To bardzo ważne, proszę zapisać jako osobny rozdział (czyli np. nie podrozdział).

\appendix
\chapter{Spis zawartości dołączonej płyty CD}\label{r:spis}
Dokładny spis zawartości towarzyszącej płytki (p. dalej). To bardzo ważne, proszę zapisać jako osobny rozdział (czyli np. nie podrozdział). Płytka CD/DVD/Blu-ray/...

Zawiera:\\
Pełną dokumentacją projektu w łatwo dającym się odczytać formacie (najlepiej pdf + źródło).
Program (w postaci źródłowej i potencjalnie umożliwiającej uruchomienie, to może oznaczać np. dostarczenie stosownych plików makefile, pomocniczych plików z danymi, opisu instalacji itp.).
Wszelkie inne dokumenty powstałe podczas zajęć (np. teksty prezentacji, teksty pracy z poprzedniego dużego punktu, itp.).

Płyta jest częścią pracy - trzeba tyle płyt co drukowanych egzemplarzy pracy. Płytkę trzeba przymocować do pracy, tak by a) nie wypadała b) dało się ją wyjąć i odczytać w komputerze :).

\begin{thebibliography}{99}
\addcontentsline{toc}{chapter}{Bibliografia}

\bibitem[DOZ]{zusdoz} ZUS, \textit{Dokumentowanie okresów zatrudnienia oraz
	wynagrodzenia}, http://www.zus.pl/files/Dokumentowanie\_okres\%C3\%B3w\_zatrudnienia.pdf

\bibitem[TFA]{tfa} Autor nieznany, \textit{The Twelve-Factor App}, https://12factor.net/

\bibitem[SDM]{govsm} Autor nieznany, \textit{Government Service Design Manual},
https://www.gov.uk/service-manual/index.html

\bibitem[MSV]{microsvc} Chris Richardson, \textit{Microservice architecture patterns and best practices},
http://microservices.io/

\bibitem[MCO]{mcobywatel} Ministerstwo Cyfryzacji, \textit{Portal Rzeczypospolitej Polskiej - Opis projektu},
https://mc.gov.pl/projekty/portal-rzeczypospolitej-polskiej/opis-projektu

\bibitem[PUE]{zuspue} ZUS, \textit{Platforma Usług Elektronicznych},
http://pue.zus.pl/

\bibitem[RDL]{redislock} Redis, \textit{Distributed locks with Redis},
https://redis.io/topics/distlock

\bibitem[RDS]{redisson} \textit{Redisson},
http://redisson.org/

\bibitem[AZK]{zookeeper} Apache, \textit{Apache ZooKeeper},
https://zookeeper.apache.org/

\bibitem[HTL]{redisbad} Martin Kleppmann, \textit{How to do distributed locking},
http://martin.kleppmann.com/2016/02/08/how-to-do-distributed-locking.html

\bibitem[CLK]{curatorlock} Apache, \textit{Apache Curator - Shared Lock},
http://curator.apache.org/curator-recipes/shared-lock.html

\end{thebibliography}

\end{document}
